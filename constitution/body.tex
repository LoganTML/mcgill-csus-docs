The Computer Science Undergraduate Society (CSUS) is only a legal body
within the compound of McGill University. This organization is strictly
a student organization and student interest group. CSUS is a non-profit
organization and receives funding based on a formula as described in the
Arts Undergraduate Society, the Science Undergraduate Society (SUS), and
the Student Society of McGill University (SSMU). CSUS may receive
external funding and is free to independently administer this funding.

\section{The Society}\label{the-society}

\subsection{Name}\label{name}

In English, the society is ``the Computer Science Undergraduate Society
of McGill University'' (CSUS), and in French it is ``L'Association des
Etudiants et Etudiantes en Informatique de L'Universite McGill'' (AEIM).

\subsection{Membership}\label{membership}

CSUS' membership is all students registered in an Undergraduate
programme in the School of Computer Science at McGill University,
including minor, liberal, and arts programs, given payment of fees
prescribed in Article 4. Any student registered in a Computer Science
Major or Computer Science Honours programme is considered a member.
Payment of society fees also grants membership. The fees for CSUS will
be determined by the Science Undergraduate Society of McGill University
(SUS) and the Arts Undergraduate Society (AUS) for their respective
constituencies.

\subsection{Purpose}\label{purpose}

The purpose of CSUS is to protect the academic rights and interests of
its constituency. Other duties of CSUS is to represent and promote the
views of its members and to implement academic, educational, cultural,
social and other programmes of interest to its members.

\subsection{Rights, Privileges and Obligations of
Members}\label{rights-privileges-and-obligations-of-members}

The privileges of the members include making use of CSUS facilities and
services, taking part in all CSUS politicking, academic, and
entertainment events, and participating in the organisation of any
politicking, academics, and events. No member is empowered to act as an
agent of CSUS unless permission to so act has been granted be the
Executive Council of CSUS.

\subsection{Finances of CSUS}\label{finances-of-csus}

The fiscal year of the Society is April 15th -- April 14th of the
following year. The accounts and inventory of CSUS, though not where the
individual items of the inventory is kept nor the source of funding,
will be publicly available at all times. Should a donor wish to do so
anonymously or privately, this may be made nonpublic, but this is not
preferable.

\subsection{Languages of the Society}\label{languages-of-the-society}

English and French are the official languages of the Society and all
meetings may be conducted in either language at the discretion of the
executive council and documents must be available in both languages on
request.

\section{Organization of the Society}\label{organization-of-the-society}

\subsection{Members of the Executive
Council}\label{members-of-the-executive-council}

The Executive Council consists of the President, Vice President (VP)
Internal, VP External, VP Finance, VP Academic, VP Administration, VP
Diversity, VP Communications, and U1 representative.

\subsubsection{Powers and Duties}\label{powers-and-duties}

The Executive Council defines all general policies of CSUS, coordinates
and administers the policies and affairs of CSUS, acts as the governing
body of CSUS, executes general assembly decisions, approves/rejects
budgets for all CSUS committees, though this task may be delegated by a
2/3 majority of the council to VP Finance, creates or dissolves CSUS
committees, determines the membership fee of membership of the society,
and appoints the Chief Returning Office. Signing powers of the Society
are exercised by at least two of the President, the VP Administration,
or the VP Internal or by any three executive council members.

\subsubsection{President}\label{president}

The President will coordinate and supervise the affairs of the Society,
call and chair over the Executive Council meetings, chair General
Assemblies, serve as an ex-officio member of all Society committees, be
the official representative of the Society, and be adminstrator of
revision control repositories and other documents.

\subsubsection{Vice-President External}\label{vice-president-external}

The Vice-President External will, in the absence of the President, be
empowered to perform any function of the President, be responsible for
maintaining links with student organizations at the university,
provincial, federal and international levels and with computer science
student societies of other universities, and be responsible for
maintaining relations with industry, government and other groups outside
University.

\subsubsection{Vice-President Internal}\label{vice-president-internal}

The Vice-President Internal will be responsible for the organization of
social, cultural and other activities for the members of CSUS, be chair
of the Social Activities Committee (SAC), maintain and promote relations
with other Faculties, Student Associations, and administrative bodies of
the University (Internal Affairs), and be responsible for the society's
facilities and equipment.

\subsubsection{Vice-President Finance}\label{vice-president-finance}

The Vice-President Finance will, in cooperation with the Executive
Council, prepare the annual budget of CSUS, which includes the actual
expenditures from the previous year, before October 15, in cooperation
with the Executive Council, manage the funds of CSUS, keep proper
financial accounts and records, and prepare a year-end financial report
by April 14th for review by an independent person at minimal cost to the
society.

\subsubsection{Vice-President Academic}\label{vice-president-academic}

The Vice-President Academic will be responsible for all educational and
curricular concerns of CSUS, whether they are internal or external to
the University, may represent a student, upon the demand of the student
in writing, in any judicial or academic or social proceedings taken
against the student by the University, or a delegate appointed by the
University, and be chair of the University Academic Committee (UAC).

\subsubsection{Vice-President
Administration}\label{vice-president-administration}

The Vice-President Administration will prepare and issue agendas and
minutes of CSUS Executive Council meetings and General Assemblies at
least three school days prior to any CSUS Executive Council meeting or
General Assembly; promote and coordinate communication within the
Society; maintain the files of the Society; ensure members of CSUS
Executive Council attend meetings and properly maintain their presence
on necessary parts of source control; be responsible to have official
minutes of the Executives and the General Assembly publicly available;
and serve as an ex-officio member of all Society committees.

\subsubsection{Vice-President Events}\label{vice-president-events}

The Vice-President of events presides over events and works towards both
the creation of new events the success of traditional events.

\subsubsection{Equity Commissioner}\label{equity-commissioner}

The Equity Commissioner (EC) promotes diversity and equity for
underrepresented groups in CSUS, acts as a liaison to other diversity
bodies both inside and outside of the university, investigates
strategies to build a supportive and inclusive community and to improve
equity, respectfulness and dialogue within the department, and is
responsible for Diversity@SOCS. The EC also commits to participate in at
least one anti-oppression/equity training or workshop.

\subsection{Vice-President
Communications}\label{vice-president-communications}

The Vice-President Communications monitors interactions between CSUS and
its constituency. This includes controlling the CSUS Facebook account,
its listserv, its Twitter, and any other methods of communication that
will ensure the student body knows CSUS activity and can provide easy
feedback.

\subsection{Representatives on CSUS}\label{representatives-on-csus}

The suggested representatives include those for U1 students, the games
society, the help desk, arts students, and science students.

\subsubsection{CSUS Help Desk
Representative}\label{csus-help-desk-representative}

The CSUS Help Desk representative will ensure the overall running of the
help desk. This includes recruiting and interviewing all tutors,
creating the schedule for the help desk tutors, and working on promotion
through course websites, facebook pages, and flyers. The representative
will also provide all tutors with a document verifying their tutoring
hours at the end of the year.

\subsubsection{U1 Representative}\label{u1-representative}

The U1 representative will facilitate event organizer, create new events
for U1 Student.

\subsection{Meetings of the Executive
Council}\label{meetings-of-the-executive-council}

Quorum for a Regular meeting of CSUS Executive Council is four members
and quorum for online interactions, where relevant, is six members. Each
member of the Executive Council has a single vote, and the president
decides the course of action in the instance of a tie. The council may
run its meetings as deemed appropriate, all of which are available to
the public.

\subsection{Powers of Assembly}\label{powers-of-assembly}

The General Assembly may make any decision, including the ratification
or rejection of any Executive Council decision and initiation of
impeachment.

\subsection{Meetings of the General
Assembly}\label{meetings-of-the-general-assembly}

\begin{enumerate}
\def\labelenumi{\arabic{enumi}.}
\item
  The executive council may call a general assembly at any time.
\item
  There are no restrictions or regulations regarding the motions or
  procedurs at the general assembly unless set by the current council.
\item
  The Quorum for a GA is 20\% of the membership.
\item
  General assemblies are held at the discretion of the current council,
  but if at least 10\% of students request a GA one must be held within
  20 days of the request.
\end{enumerate}

\subsection{Electoral Officer}\label{electoral-officer}

The Chief Returning Officer or the current president is responsible for
administering elections. The CRO may not be a candidate in any Society
election. The Executive Council elects a new CRO if the CRO decides to
run for a position.

\subsection{Referenda}\label{referenda}

\begin{enumerate}
\def\labelenumi{\arabic{enumi}.}
\item
  A referendum is initiated by petition of 10\% of the society or by 2/3
  majority on council.
\item
  Referendums must be publicly available 6 days before vote and all
  details of the voting location, hours, etc. must also be posted.
\item
  The Referendum shall be considered valid only if a minimum of 15\%
  CSUS members vote.
\item
  Referenda are passed by a majority of votership.
\item
  Referenda overrule decisions by council or by general assembly.
\item
  Referenda, including changes to this constitution, may be performed by
  a unanimous vote by council at any time, but referenda made in this
  manner must be announced to the membership so that it may be reviewed
  and a petition process begun if desired.
\end{enumerate}

\subsection{Impeachment}\label{impeachment}

Council members are impeached by 2/3 vote of executive council or by a
25\% vote of constituency.

\section{Elections}\label{elections}

\subsection{Eligible Voters and
Candidates}\label{eligible-voters-and-candidates}

\begin{enumerate}
\def\labelenumi{\arabic{enumi}.}
\item
  CSUS runs elections for the President, VP Internal, VP External, VP
  Academic, and VP Administration.
\item
  A U1 Representative, which can be more than one representative at the
  choice of the elected representative, is elected by the regular
  members who are first year computer science students in the first
  month of the school year, including those who have just transferred.
\item
  Any student of McGill University may run for CSUS positions.
\end{enumerate}

\subsection{Procedures}\label{procedures}

Elections are held between the first and last days of March. Vacancies
outside of this period may be filled at the discretion of the CRO,
president, or VP admin so long as council majority votes in agreement.
Elections can also be initiated and administered collectively by the
executive committee by a 2/3 vote to fill vacancies and such elections
will run for fifteen days starting from the day of the vote.

\subsection{Terms of Office}\label{terms-of-office}

The terms for members of the Executive Council are from the April 15 and
last one year excepting the U1 rep, who starts on the day of election or
selection.

\subsection{Order of Succession}\label{order-of-succession}

If the need for succession arises, another executive may assume his or
her duties by a 2/3 election by the executive council or, should a 2/3
vote not be reached within 5 days, a reelection from the first day at
the end of those 5 days until 10 days later.

\section{Bylaws}\label{bylaws}

\subsection{Nomination Rules}\label{nomination-rules}

\begin{enumerate}
\def\labelenumi{\arabic{enumi}.}
\item
  The CRO posts a list of positions open to nomination and election
  along with a time for the opening and closing of the nominations.
\item
  All nominations must be signed by ten students eligible to vote
  excepting for the U1 elections, for which five students must nominate.
\item
  The CRO validates the submitted nominations and publicises them within
  twelve hours of the closing of the nominations.
\end{enumerate}

\subsection{Campaigning Rules}\label{campaigning-rules}

The CRO may designate rules for the campaigning period but can be
overruled by a 2/3 vote by the existing council.

\subsection{Balloting Rules}\label{balloting-rules}

Ballots are counted as soon as practicable after the closing of the
polls under the supervision of the CRO and those whom the CRO designates
to assist him. All complaints, protest or petitions for a recount must
be made to the CRO no later than three school days following the closing
of the polls. The CRO shall invalidate the election at his discretion
with a 2/3 vote of council. Ties are broken by a majority vote in the
current executive council.

\subsection{Bylaw Amendments}\label{bylaw-amendments}

Amendments to the by-laws may be made at any meeting of CSUS Executive
Council and must be approved by two-thirds (2/3) of those present and
voting.
