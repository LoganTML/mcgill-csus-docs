\section*{Preamble}\label{preamble}

The Computer Science Undergraduate Society (CSUS) is only a legal body within the compound of McGill University. This organization is strictly a student organization and student interest group. The CSUS is a non-profit departmental association of the Science Undergraduate Society and the Arts Undergraduate Society and receives funding based on a formula as described in the Arts Undergraduate Society, Science Undergraduate Society and the Students' Society of McGill University, as well as organizations external to McGill University.

\section*{Terms and Definitions}\label{terms-and-definitions}

\begin{enumerate}
\def\labelenumi{\arabic{enumi}.}
\tightlist
\item
  McGill University will be herein referred to as the ``University''.
\item
  The School of Computer Science at McGill University will be herein
  referred to as the ``SOCS''.
\item
  The Arts Undergraduate Society of McGill University will be herein
  referred to as the ``AUS''.
\item
  The Science Undergraduate Society of McGill University will be herein
  referred to as the ``SUS''.
\item
  The Students' Society of McGill University will be herein referred to
  as the ``SSMU''.
\item
  The members of the CSUS are also referred to as the ``constituency''
  and these two terms are equivalent.
\end{enumerate}

\part{The Society}\label{the-society}

\section{Name}\label{name}

The official name of the Society, in EngIish, shall be the ``Computer
Science Undergraduate Society of McGill University'', and in French,
``L'Association des Etudiants et Etudiantes en Informatique de
L'Universite McGill'', herein referred to as CSUS and AEIM respectively.

\section{Membership}\label{membership}

\begin{enumerate}
\def\labelenumi{\arabic{enumi}.}
\tightlist
\item
  Membership of the CSUS shall be all students registered in an
  Undergraduate programme offered by, or jointly with the SOCS, subject
  to payment of fees prescribed in Article \ref{society-fees}.
\end{enumerate}

\section{Purpose}\label{purpose}

\begin{enumerate}
\def\labelenumi{\arabic{enumi}.}
\tightlist
\item
  The purpose of the CSUS is to protect the academic rights and
  interests of its constituency.
\item
  Other duties of the CSUS is to represent and promote the views of its
  members and to implement academic, educational, cultural, social and
  other programmes of interest to its members.
\end{enumerate}

\section{Society Fees}\label{society-fees}

The fees for the CSUS will be determined by the SUS and AUS.

\section{Rights, Privileges and Obligations of
Members}\label{rights-privileges-and-obligations-of-members}

\begin{enumerate}
\def\labelenumi{\arabic{enumi}.}
\tightlist
\item
  The rights of the members shall include the following:
\end{enumerate}

\begin{enumerate}
\def\labelenumi{(\alph{enumi})}
\item
  the right to vote in the CSUS general elections and referenda;
\item
  the right to attend General Assemblies and Executive Council meetings
  of the CSUS;
\item
  the right to initiate referenda or General Assemblies according to
  Articles \ref{meetings-of-the-general-assembly} and \ref{referenda};
\item
  the right to speak at any General Assembly;
\item
  the right to nominate candidates for CSUS elections according to the
  Electoral bylaws.
\end{enumerate}

\begin{enumerate}
\def\labelenumi{\arabic{enumi}.}
\setcounter{enumi}{1}
\tightlist
\item
  The privileges of the members shall include:
\end{enumerate}

\begin{enumerate}
\def\labelenumi{(\alph{enumi})}
\tightlist
\item
  holding office within the CSUS subject to the qualifications as
  specified in Article \ref{eligible-voters-procedures-and-candidates};
\item
  making use of the CSUS facilities and services.
\end{enumerate}

\begin{enumerate}
\def\labelenumi{\arabic{enumi}.}
\setcounter{enumi}{2}
\item
  The obligations of all members of the CSUS will be to conform to the
  CSUS Constitution, regulations and bylaws.
\item
  No member is empowered to make purchase in the name of the CSUS or to
  financially obligate the CSUS in any way, until such permission has
  been granted by the Executive Council of the CSUS.
\item
  No member is empowered to act as an agent of the CSUS unless
  permission to so act has been granted by the Executive Council of the
  CSUS.
\end{enumerate}

\section{Finances of the CSUS}\label{finances-of-the-csus}

\begin{enumerate}
\def\labelenumi{\arabic{enumi}.}
\tightlist
\item
  The fiscal year of the CSUS shall from May 1st to April 30th of the following year.
\item
  The budget of the CSUS for the current fiscal year will be presented
  to the AUS and the SUS no later than October 1st.
\item
  The accounts of the CSUS shall be maintained according to standard
  accounting practice and shall be made available to the University
  auditors or any student on demand.
\end{enumerate}

\section{Languages of the CSUS}\label{languages-of-the-csus}

\begin{enumerate}
\def\labelenumi{\arabic{enumi}.}
\tightlist
\item
  English and French are the official languages of the CSUS.
\item
  At all meetings of the CSUS, members may use either official
  languages.
\item
  Resolutions of the may be adopted in either or both official
  languages.
\item
  Documents may be obtained from the CSUS in any Particular language
  upon demand.
\end{enumerate}

\part{Organization of the
CSUS}\label{organization-of-the-csus}

\section{Members of the Executive
Council}\label{members-of-the-executive-council}

\begin{enumerate}
\def\labelenumi{\arabic{enumi}.}
\tightlist
\item
  The Executive Council shall consist of elected and appointed members.
  The elected members are:
\end{enumerate}

\begin{enumerate}
\def\labelenumi{(\alph{enumi})}
\item
  the President;
\item
  the Vice-President Internal;
\item
  the Vice-President External;
\item
  the Vice-President Academic;
\item
  the Vice-President Administration;
\end{enumerate}

  And the appointed members are: 

\begin{enumerate}
\def\labelenumi{(\alph{enumi})}
\setcounter{enumi}{5}
\item
  the Vice-President Finance;
\item
  the Vice-President Communications;
\item
  the Vice-President Events;
\item
  the Vice-President Arts;
\item
  the Vice-President Science;
\item
  the Equity Commissioner;
\item
  the Representatives appointed to the Executive Council.
\end{enumerate}

\begin{enumerate}
\def\labelenumi{\arabic{enumi}.}
\setcounter{enumi}{1}
\item
  Elected members of the Executive Council
  shall be elected according to the provisions of Part \ref{elections} (Elections).
\item
  Appointed members of the Executive Council
  shall be appointed according to provisions set by the elected members of
  a new Executive Council.
\item
  Members of the Executive Council must be members of the CSUS for the
  entire duration of their mandate. No exception shall be granted.
\item
  No member of the Executive Council shall receive financial
  remuneration for the fulfillment of their mandate.
\end{enumerate}

\section{Powers and Duties}\label{powers-and-duties}

\begin{enumerate}
\def\labelenumi{\arabic{enumi}.}
\tightlist
\item
  The Executive Council shall:
\end{enumerate}

\begin{enumerate}
\def\labelenumi{(\alph{enumi})}
\item
  define all general policies of the CSUS;
\item
  coordinate and administer the policies, activities and other day to
  day affairs of the CSUS;
\item
  act as the governing body of the CSUS, empowered to make all decisions
  and take responsible and required actions on behalf of the CSUS;
\item
  be held accountable to General Assembly and ensure the execution of
  General Assembly decisions;
\item
  approve or reject budgets for the CSUS, This task may be delegated
  to the Vice-President Finance with the approval of 2/3 majority of
  members present and voting;
\item
  uphold the Constitution, regulations, policies and bylaws of the CSUS;
\item
  appoint the Chief Returning Officer;
\item
  appoint the appointed members of the executive council in consultation with any member of CSUS interested in taking part in the selection process;
\item
  appoint Representatives subject to the procedures described in Article \ref{representatives-of-the-csus} and in consultation with any member of CSUS interested in taking part in the selection process.
\end{enumerate}

\begin{enumerate}
\def\labelenumi{\arabic{enumi}.}
\setcounter{enumi}{1}
\tightlist
\item
  Signing powers of the CSUS shall be exercised by the following, any
  two of whom shall be required for any given transaction:
\end{enumerate}

\begin{enumerate}
\def\labelenumi{(\alph{enumi})}
\tightlist
\item
  the President;
\item
  the Vice-President Finance;
\item
  the Vice-President Administration;
\item
  the Vice-President External;
\item
  the Vice-President Internal;
\item
  the Vice-President Academic;
\end{enumerate}
  or any three members of the Executive
  Council who are not representatives.

\begin{enumerate}
\def\labelenumi{\arabic{enumi}.}
\setcounter{enumi}{2}
\item
  Each member of the Executive Council shall prepare and submit to the
  Executive Council a year-end report no later than April 14th.
\item
  A member of the Executive Council shall cease to remain in office upon
  acceptance of their letter of resignation, upon their impeachment, or at the end of their term of office.
\end{enumerate}

\section{President}\label{president}

The President shall:

\begin{enumerate}
\def\labelenumi{\arabic{enumi}.}
\item
  Coordinate and supervise the affairs of the CSUS;
\item
  Call and chair over the Executive Council meetings;
\item
  Chair General Assemblies;
\item
  Be the official representative of the CSUS.
\end{enumerate}

\section{Vice-President
External}\label{vice-president-external}

The Vice-President External shall:

\begin{enumerate}
\def\labelenumi{\arabic{enumi}.}
\item
  In the absence of the President, be empowered to perform any function
  of the President;
\item
  Be responsible for maintaining links with student organizations at the
  university, provincial, federal and international levels and with
  computer science student societies of other universities;
\item
  Be responsible for maintaining relations with industry, government and
  other groups outside the University.
\end{enumerate}

\section{Vice-President
Internal}\label{vice-president-internal}

The Vice-President Internal shall:

\begin{enumerate}
\def\labelenumi{\arabic{enumi}.}
\item
  Be responsible for the organization of social, cultural and other
  activities for the members of the CSUS;
\item
  Maintain and promote relations with other Faculties, Student
  Associations and administrative bodies of the University (Internal
  Affairs).
\item
  Be responsible for the CSUS's facilities and equipment.
\item
  Direct, support, and coordinate the activities of the Vice-President
  Events, Vice-President Arts, and Vice-President Science and carry out
  their duties when they are not able to do so.
\end{enumerate}

\section{Vice-President Finance}\label{vice-president-finance}

The Vice-President Finance shall:

\begin{enumerate}
\def\labelenumi{\arabic{enumi}.}
\item
  In cooperation with the Executive Council, prepare the annual budget
  of the CSUS, which shall include the actual expenditures from the
  previous year, before October 15;
\item
  In cooperation with the Executive Council, manage the funds of the
  CSUS;
\item
  Keep proper financial accounts and records;
\item
  Prepare a year-end financial report by April 14th.
\item
  Present a complete semester-end financial report that shall be made
  public and shall be readily available to the constituents.
\end{enumerate}

\section{Vice-President
Academic}\label{vice-president-academic}

The Vice-President Academic shall:

\begin{enumerate}
\def\labelenumi{\arabic{enumi}.}
\item
  Be responsible for all educational and curricular concerns of the
  CSUS, whether they be internal or external to the University.
\item
  Represent any student, upon the request of the student in writing, in
  any judicial or academic or social proceedings taken against the
  student by the University, or a delegate appointed by the University.
\end{enumerate}

\section{Vice-President
Administration}\label{vice-president-administration}

The Vice-President Administration shall:

\begin{enumerate}
\def\labelenumi{\arabic{enumi}.}
\item
  Be responsible for preparing and issuing agendas and minutes of the
  CSUS Executive Council meetings and General Assemblies at least three
  (3) school days prior to any CSUS Executive Council meeting or General
  Assembly;
\item
  Promote and coordinate communication within the CSUS;
\item
  Maintain the files of the CSUS;
\item
  Ensure members of the CSUS Executive Council attend meetings;
\item
  Be responsible to have official minutes of the Executives and the
  General Assembly readily available on demand.
\end{enumerate}

\section{Vice-President Events}\label{vice-president-events}

The Vice-President Events shall:

\begin{enumerate}
\def\labelenumi{\arabic{enumi}.}
\item
  Preside over the planning and delivery of activities organized by the
  CSUS;
\item
  Promote the above activities to members of the CSUS;
\item
  Assess feedback and explore possibilities for new activities to better
  engage the member of the CSUS.
\end{enumerate}

\section{Vice-President Communications}\label{vice-president-communications}

The Vice-President Communications shall:

\begin{enumerate}
\def\labelenumi{\arabic{enumi}.}
\item
  Create and maintain the mechanisms by which the CSUS communicates with
  its constituents;
\item
  Monitor interactions between the CSUS and its constituency on social
  media, advertisements, and anywhere else constituents may choose to
  communicate with the CSUS;
\item
  Inform the constituency of activities and news of the CSUS.
\end{enumerate}

\section{Vice-President
Arts}\label{vice-president-arts}

The Vice-President Arts shall:

\begin{enumerate}
\def\labelenumi{\arabic{enumi}.}
\item
  Attend every meeting of the AUS Legislative Council;
\item
  Act as a liaison between the AUS and the CSUS;
\item
  Promote and represent the interests of constituents in the Faculty of
  Arts.
\end{enumerate}

\section{Vice President Science}\label{vice-president-science}

The Vice-President Science shall:

\begin{enumerate}
\def\labelenumi{\arabic{enumi}.}
\item
  Attend every meeting of the SUS General Council;
\item
  Act as a liaison between the SUS and the CSUS;
\item
  Promote and represent the interests of constituents in the Faculty of
  Science.
\end{enumerate}

\section{Equity Commissioner}\label{equity-commissioner}

The Equity Commissioner shall:

\begin{enumerate}
\def\labelenumi{\arabic{enumi}.}
\item
  Promote diversity and equity for underrepresented groups in CSUS;
\item
  Act as a liaison to other diversity bodies both inside and outside of
  the university;
\item
  Investigates strategies to build a supportive and inclusive community;
\item
  Improve equity, respectfulness and dialogue within the dePartment;
\item
  Be responsible for the Diversity@SOCS group;
\item
  Participate in at least one (1) anti-oppression / equity training or
  workshop.
\end{enumerate}

\section{Representatives of the
CSUS}\label{representatives-of-the-csus}

\begin{enumerate}
\def\labelenumi{\arabic{enumi}.}
\item
  The Executive Council may appoint up to twelve (12) representatives to
  provide representation on the Executive Council of the various
  constituencies of the CSUS' members.
\item
  The Executive Council must appoint at least one (1) U1 Representative
  and one (1) Help Desk Representative.
\item
  The Executive Council must consider appointing a representative for a
  certain interest upon receipt of a petition of 10 members of the CSUS.
  The petition must contain the words, "We, the undersigned, request
  the appointment of a representative of CSUS for \_\_".
\end{enumerate}

\subsection{U1 Representative}\label{u1-representative}

The U1 Representative shall:

\begin{enumerate}
\def\labelenumi{\arabic{enumi}.}
\item
  Act as liaison between the U1 students and the Executive Council;
\item
  Represent the views of the U1 students at meetings of the Executive
  Council;
\item
  Hold a meeting of the U1 students when necessary.
\end{enumerate}

\subsection{Help Desk Representative}\label{help-desk-representative}

The Help Desk Representative shall:

\begin{enumerate}
\def\labelenumi{\arabic{enumi}.}
\item
  Act as liaison between the Help Desk tutors and the Executive Council;
\item
  Find, interview, and appoint tutors to the Help Desk;
\item
  Create and maintain the schedule for tutors;
\item
  Obtain and deliver to all tutors who have successfully completed their
  appointment a letter attesting to the number of hours they have
  contributed at the end of each academic semester;
\item
  Organize and deliver rewards for the tutors to recognize their
  contribution.
\end{enumerate}

\section{Meetings of the Executive
Council}\label{meetings-of-the-executive-council}

\begin{enumerate}
\def\labelenumi{\arabic{enumi}.}
\item
  The CSUS Executive Council shall hold meetings at least once every two
  weeks while classes are in session.
\item
  Quorum for a Regular meeting of the CSUS Executive Council shall be
  four (4) members.
\item
  Each member of the Executive Council shall have a single vote, except
  for the President who shall only vote to break a tie.
\item
  In case of an emergency, the President may call a special meeting at
  any time, provided that all executives are present, or upon the
  signing of a waiver of notice by all executives.
\end{enumerate}

\section{Powers of Assembly}\label{powers-of-assembly}

\begin{enumerate}
\def\labelenumi{\arabic{enumi}.}
\item
  The General Assembly may make any decision, including the ratification
  or rejection of any Executive Council decision.
\item
  The General Assembly may also impeach a (the) member(s) of the
  Executive Council as prescribed under Article \ref{impeachment}.
\end{enumerate}

\section{Meetings of the General
Assembly}\label{meetings-of-the-general-assembly}

\begin{enumerate}
\def\labelenumi{\arabic{enumi}.}
\item
  The General Assembly may be called by a resolution of the Executive
  Councilor or by a petition signed by at least 50
  members of the Society.
\item
  A notice announcing the meeting of the General Assembly shall be
  posted at least five (5) teaching days prior to the convening of the
  General Assembly. The notice must be posted in areas where the maximum
  number of Computer Science students will be able to see.
\item
  The Quorum for the General Assembly shall be the lower standard of measure between ten percent (10\%) of the members of CSUS or 50 members of the CSUS.
\item
  A General Assembly of the CSUS shall be held within twenty (20) school
  days upon receipt of a petition of 10 members of the CSUS. The
  petition must contain the words, ``We, the undersigned, request a
  General Assembly of the CSUS''.
\end{enumerate}

\section{Electoral Officer}\label{electoral-officer}

\begin{enumerate}
\def\labelenumi{\arabic{enumi}.}
\item
  The Chief Returning Officer (CRO) shall be responsible for all aspects
  of the administration of CSUS elections and referenda according to the
  Electoral bylaws of the CSUS.
\item
  The CRO may not be a candidate in any Society election; if she or he wishes
  to be a candidate for any elected Society position, the CRO shall
  resign from the position of CRO. The Executive Council shall then
  elect a new CRO subject to Article \ref{filling-of-vacancies}.
\item
  The CRO may be removed from office for dereliction of duties as
  specified in Article \ref{impeachment}.
\end{enumerate}

\section{Referenda}\label{referenda}

\begin{enumerate}
\def\labelenumi{\arabic{enumi}.}
\item
  A Referendum may be initiated by a resolution of the Executive Council
  or a petition signed by at least twenty percent (20\%) of the members
  of the Society. The requirements for impeachment are described in
  Article \ref{impeachment}.
\item
  Notice of the referendum question, voting location and voting hours
  must be posted no less than six (6) school days before the vote is to
  take place. All notices must be posted in highly public area,
  especially areas frequented by CS students.
\item
  The referendum shall be held according to the Electoral bylaws and
  procedures stipulated in Part \ref{elections} (Elections) of the constitution of
  the CSUS.
\item
  The Referendum shall be considered valid only if a minimum of ten
  percent (10\%) of the CSUS members vote.
\item
  A simple majority is required to pass a referendum. The requirements
  for constitutional amendments are stated in Article \ref{constitutional-amendments}.
\item
  The result of a Referendum shall be binding on the Society and take
  precedence over decisions of the Executive Council and the General
  Assembly.
\end{enumerate}

\section{Impeachment}\label{impeachment}

\begin{enumerate}
\def\labelenumi{\arabic{enumi}.}
\tightlist
\item
  A member of the Executive Council may be removed by way of referendum
  initiated by either:
\end{enumerate}

\begin{enumerate}
\def\labelenumi{(\alph{enumi})}
\tightlist
\item
  two-thirds (2/3) majority vote of the Executive Council;
\item
  a petition signed by 100 members of the
  CSUS.
\end{enumerate}

\begin{enumerate}
\def\labelenumi{\arabic{enumi}.}
\setcounter{enumi}{1}
\tightlist
\item
  The CRO may be relieved of his or her duties by a two-thirds (2/3)
  vote of the Executive Council.
\end{enumerate}

\part{Elections}\label{elections}

\section{Eligible Voters, Procedures and
Candidates}\label{eligible-voters-procedures-and-candidates}

\begin{enumerate}
\def\labelenumi{\arabic{enumi}.}
\item
  All undergraduate students recognized as being in a program offered by
  the SOCS at the start time of a ballot are considered to be eligible
  voters.
\item
  The Chief Returning Officer may not be candidate in any election of
  the CSUS.
\item
  Elections for all elected members of the Executive Council shall be
  held between March 1st and March 31st.
\item
  The elections shall be held according to the Bylaws of the CSUS.
\end{enumerate}

\section{Filling of Vacancies}\label{filling-of-vacancies}

In the event of a vacancy of an elected position on the Executive
Council, the Executive Council shall instruct the CRO to hold a
by-election according to the Electoral bylaws of the CSUS.

\section{Terms of Office}\label{terms-of-office}

\begin{enumerate}
\def\labelenumi{\arabic{enumi}.}
\tightlist
\item
  The terms of office for all members of the Executive Council shall
  begin on May 1st and end on April 30th of the following year.
\end{enumerate}

\section{Order of Succession}\label{order-of-succession}

\begin{enumerate}
\def\labelenumi{\arabic{enumi}.}
\tightlist
\item
  For the purpose of continuity, in the case of prolonged absence,
  illness, resignation, impeachment or death of an executive, the
  following options are available:
\end{enumerate}

\begin{enumerate}
\def\labelenumi{(\alph{enumi})}
\tightlist
\item
  Another executive may assume the duties;
\item
  A by-election subject to the Election By-laws.
\end{enumerate}

\begin{enumerate}
\def\labelenumi{\arabic{enumi}.}
\setcounter{enumi}{1}
\tightlist
\item
  In the event of prolonged absence or illness, resignation, impeachment
  or death of the President of the CSUS, the order of succession shall
  be as follows:
\end{enumerate}

\begin{enumerate}
\def\labelenumi{(\alph{enumi})}
\item
  Vice-President External;
\item
  Vice-President Internal;
\item
  Vice-President Administration;
\item
  Vice-President Academic;
\item
  Vice-President Finance;
\item
  Vice-President Events;
\item
  Vice-President Communications;
\item
  Vice-President Science;
\item
  Vice-President Arts;
\item
  The representatives in order of date of appointment.
\end{enumerate}

\begin{enumerate}
\def\labelenumi{\arabic{enumi}.}
\setcounter{enumi}{2}
\tightlist
\item
  By definition, a person is incapacitated when he/she is no longer able
  to perform the duties as specified in the CSUS's Constitution, and is
  unable to continue their mandate.
\item
  The C.R.O. is listed, in the event none of the members of the Executive
  Council is able to assume the office of President, shall be the acting
  President until such time as an election or by-election is held to fill
  all vacancies as prescribed by the CSUS's Constitution and Electorate
  Bylaws.
\end{enumerate}

\part{The Constitution}\label{the-constitution}

\section{Superseding Clause}\label{superseding-clause}

This Constitution supersedes and repeals all previous Constitutions of
the CSUS.

\section{Constitutional
Amendments}\label{constitutional-amendments}

\begin{enumerate}
\def\labelenumi{\arabic{enumi}.}
\item
  The Constitution of the society may only be amended by a referendum in
  accordance to Article \ref{referenda} with a majority of two-thirds (2/3) of the
  members voting in favour.
\item
  All amendments of the constitution must be approved by the SUS and AUS
  in accordance with their procedures for amendments.
\end{enumerate}

\section{Coming into force}\label{coming-into-force}

\begin{enumerate}
\def\labelenumi{\arabic{enumi}.}
\tightlist
\item
  The Constitution shall come into force March 3, 2016.
\end{enumerate}

\part{Bylaws}\label{bylaws}
\setcounter{section}{0}
\renewcommand\thesection{\Alph{section}}

\section{Bylaw Amendments}\label{bylaw-amendments}

Amendments to the by-laws may be made at any meeting of the CSUS
Executive Council and must be approved by two-thirds (2/3) of those
present and voting.

\section{Electoral Bylaws}\label{electoral-bylaws}

\subsection{Nomination Rules}\label{nomination-rules}

\begin{enumerate}
\def\labelenumi{\arabic{enumi}.}
\item
  The CRO shall post, at an appropriate time, giving notice to all
  students, a list of positions open to nomination and election along
  with a time for the opening and closing of the nominations.
\item
  The closing date for the nominations will be no later than five (5)
  school days before the elections.
\item
  All nominations must contain the words ``We, the
  undersigned, nominate \_ for the position of \_ for the 20(n)-20(n+1)
  academic year.''
\item
  With respect to the general elections, all nominations must be signed
  by ten (10) students eligible to vote according to Article \ref{eligible-voters-procedures-and-candidates}.
\item
  All nominations must be presented to the CRO before the closing dead
  line established by the CRO.
\item
  If at the close of nominations, any position is such that it would
  result in a vacancy or acclamation, the CRO shall re-open nominations
  for one (1) additional teaching day.
\item
  The CRO shall validate the nominations and publicize them within
  twelve (12) hours of the closing of the nominations.
\end{enumerate}

\subsection{Campaigning Rules}\label{campaigning-rules}

\begin{enumerate}
\def\labelenumi{\arabic{enumi}.}
\item
  The CRO shall post the date, time and location of the elections at
  least five (5) school days before the elections.
\item
  The campaign period for all candidates will commence one week before
  the election and will end at 23:00 hrs. of the day before the
  election.
\item
  The posting of notices and signs must conform to the general rules of
  the University concerning the placing of such materials on the
  University premises.
\item
  The CRO may designate rules from time to time with respect to the
  placing of such materials in the University.
\item
  The CRO may set a maximum spending limit of which the CSUS will
  reimburse each candidate a predetermined amount. (Ensure that all
  candidates get to spend the same amount).
\item
  All receipts (except for the use of an automobile and public
  transport) must be turned over to the CRO before the end of the voting
  period. The CRO has the right to refuse a (all) receipt(s) that are
  not deemed to be proper, official or otherwise.
\end{enumerate}

\subsection{Balloting Rules}\label{balloting-rules}

\begin{enumerate}
\def\labelenumi{\arabic{enumi}.}
\tightlist
\item
  The CRO shall be responsible for appointing polling clerks.
\end{enumerate}

\begin{enumerate}
\def\labelenumi{(\alph{enumi})}
\tightlist
\item
  Polling clerks shall be paid the minimum wage/hour, or a predetermined
  amount not to be less that the (hours * minimum wage) formula.
\end{enumerate}

\begin{enumerate}
\def\labelenumi{\arabic{enumi}.}
\setcounter{enumi}{1}
\item
  Balloting shall be done in accordance to Article \ref{eligible-voters-procedures-and-candidates}.
\item
  Before polls open, the CRO shall cast his vote and seal it under guard
  of the President and Vice-President Administration of the Society. The
  vote of the CRO shall only be opened in the case of a tie in which
  case it will only be used to break a Particular tie(s).
\end{enumerate}

\subsection{Count, Recount and
Protests}\label{count-recount-and-protests}

\begin{enumerate}
\def\labelenumi{\arabic{enumi}.}
\item
  Ballots shall be counted as soon as practicable after the closing of
  the polls under the supervision of the CRO and those whom the CRO
  designate to assist him.
\item
  No ballot shall be counted in the presence of less than two persons.
\item
  Ballot shall be rejected if they are deemed spoiled by the CRO.
\item
  When the counting of the ballots has been completed, the CRO will
  announce the results to the candidates. The results shall then be
  posted publicly.
\item
  All complaints, protest or petitions for a recount must be made to the
  CRO no later than three (3) school days following the closing of the
  polls.
\end{enumerate}

\subsection{Invalidation}\label{invalidation}

The CRO shall invalidate the election if upon investigation it is
evident that there was a gross violation of these bylaws such as to:

\begin{enumerate}
\def\labelenumi{\arabic{enumi}.}
\item
  disenfranchise eligible voters;
\item
  permit ineligible persons to vote;
\item
  mislead voters in their choice;
\item
  interfere with voting on election days.
\item
  groups running as a Party (be it official or unofficial). If a group
  of candidates decide to run as a collective, then the CRO shall give
  notice that such action is illegal, as well, as advertisement with the
  names of collective candidates.
\end{enumerate}
