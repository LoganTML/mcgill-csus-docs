The Computer Science Undergraduate Society (CSUS) is only a legal body
within the compound of McGill University. This organization is strictly
a student organization and student interest group. CSUS is a non-profit
organization and receives funding based on a formula as described in the
Science Undergraduate Society (SUS) and the Student Society of McGill
University (SSMU). CSUS may receive external funding and is free to
independently administer this funding.

\section{The Society}\label{the-society}

\subsection{Name}\label{name}

In English, the society is ``the Computer Science Undergraduate Society
of McGill University,'' and in French it is ``L'Association des
Etudiants et Etudiantes en Informatique de L'Universite McGill''. These
will be referred to as CSUS and AEIM respectively throughout this
document.

\subsection{Membership}\label{membership}

CSUS' membership is all students registered in an Undergraduate
programme in the School of Computer Science at McGill University, given
payment of fees prescribed in Article 4. Any student registered in a
Computer Science Major or Computer Science Honours programme is
considered a member. Students may be granted membership by the Executive
Council subject to Article 4 (payment).

\subsection{Purpose}\label{purpose}

The primary purpose of CSUS is to protect the academic rights and
interests of its constituency. Other duties of CSUS is to represent and
promote the views of its members and to implement academic, educational,
cultural, social and other programmes of interest to its members.

\subsection{Society Fees}\label{society-fees}

The fees for CSUS will be determined by the Science Undergraduate
Society of McGill University (Hereafter referred to as SUS). Additional
fees specific to CSUS may be added by referendum.

\subsection{Rights, Privileges and Obligations of
Members}\label{rights-privileges-and-obligations-of-members}

The privileges of the members include making use of CSUS facilities and
services, taking part in all CSUS politicking, academic, and
entertainment events, and participating in the organisation of any
politicking, academics, and events. It is an obligation of all CSUS
members to conform to CSUS Constitution, regulations and bylaws. No
member is empowered to act as an agent of CSUS unless permission to so
act has been granted be the Executive Council of CSUS.

\subsection{Finances of CSUS}\label{finances-of-csus}

The fiscal year of the Society is April 15th -- April 14th of the
following year. The accounts and inventory of CSUS, though not where the
individual itmes of the inventory is kept, will be publicly available at
all times.

\subsection{Languages of the Society}\label{languages-of-the-society}

English and French are the official languages of the Society and all
meetings may be conducted in either language at the discretion of the
executive council and documents must be available in both languages on
request.

\section{Organization of the Society}\label{organization-of-the-society}

\subsection{Members of the Executive
Council}\label{members-of-the-executive-council}

The Executive Council consists of the President, Vice President (VP)
Internal, VP External, VP Finance, VP Academic, VP Administration, VP
Diversity, VP Sustainability, VP Events, and U1 representative.

\subsection{Powers and Duties}\label{powers-and-duties}

The Executive Council defines all general policies of CSUS, coordinates
and administers the policies and affairs of CSUS, acts as the governing
body of CSUS, executes general assembly decisions, approves/rejects
budgets for all CSUS committees, though this task may be delegated by a
2/3 majority of the council to VP Finance, creates or dissolves CSUS
committees, determines the membership fee of membership of the society,
and appoints the Chief Returning Office. Signing powers of the Society
shall are exercised by at least two of the President, the VP Finance,
the VP Administration, and the VP Academic or by any four executive
council members.

\subsection{President}\label{president}

The President will coordinate and supervise the affairs of the Society,
call and chair over the Executive Council meetings, chair General
Assemblies, serve as an ex-officio member of all Society committees, be
the official representative of the Society, and be adminstrator of
revision control repositories and other documents.

\subsection{Vice-President External}\label{vice-president-external}

The Vice-President External will, in the absence of the President, be
empowered to perform any function of the President, be responsible for
maintaining links with student organizations at the university,
provincial, federal and international levels and with computer science
student societies of other universities, and be responsible for
maintaining relations with industry, government and other groups outside
University.

\subsection{Vice-President Internal}\label{vice-president-internal}

The Vice-President Internal will be responsible for the organization of
social, cultural and other activities for the members of CSUS, be chair
of the Social Activities Committee (SAC), maintain and promote relations
with other Faculties, Student Associations, and administrative bodies of
the University (Internal Affairs), and be responsible for the society's
facilities and equipment.

\subsection{Vice-President Finance}\label{vice-president-finance}

The Vice-President Finance will, in cooperation with the Executive
Council, prepare the annual budget of CSUS, which includes the actual
expenditures from the previous year, before October 15, in cooperation
with the Executive Council, manage the funds of CSUS, keep proper
financial accounts and records, and prepare a year-end financial report
by April 14th for review by an independent person at minimal cost to the
society.

\subsection{Vice-President Academic}\label{vice-president-academic}

The Vice-President Academic will be responsible for all educational and
curricular concerns of CSUS, whether they are internal or external to
the University, may represent a student, upon the demand of the student
in writing, in any judicial or academic or social proceedings taken
against the student by the University, or a delegate appointed by the
University, and be chair of the University Academic Committee (UAC).

\subsection{Vice-President
Administration}\label{vice-president-administration}

The Vice-President Administration will be responsible for preparing and
issuing agendas and minutes of CSUS Executive Council meetings and
General Assemblies at least three school days prior to any CSUS
Executive Council meeting or General Assembly; promote and coordinate
communication within the Society; maintain the files of the Society;
ensure members of CSUS Executive Council attend meetings and properly
maintain their presence on necessary parts of source control; be
responsible to have official minutes of the Executives and the General
Assembly publicly available; and serve as an ex-officio member of all
Society committees;

\subsection{U1 Representative}\label{u1-representative}

The U1 Representative will act as liaison between the U1 students and
the Executive Council; represent the views of the Ul students at
meetings of the Executive Council; hold a meeting of the Ul students
when necessary; serve on the University Academic Committee (UAC) as
Vice-Chairperson; serve as the Vice-Chair of the Executive Council.

\subsection{Vice-President
Sustainability}\label{vice-president-sustainability}

The Vice-President Sustainability will be responsible for ensuring
events and activities of CSUS are carried out in a sustainble fashion,
determining how to maximise sustainability while minising cost, be
responsible for promoting sustainability amongst the constituency, and
create and promote sustainability events in conjunction with
Vice-President Events.

\subsection{Vice-President Diversity}\label{vice-president-diversity}

The Vice-President Diversity will promote diversity and equity for 
underrepresented groups in CSUS (people of colour, LGBTQ, women,
 indigenous people, people with (dis)abilities, etc.), act as a liaison to other 
diversity bodies both inside and outside of the university, investigate issues 
on diversity and strategies to build a supportive and inclusive community 
within the department, and otherwise improve equity, respectfulness and 
dialogue amongst the constituency. By accepting this role, VP Diversity also 
commits to participate in at least one anti-oppression workshop. It is strongly 
recommended that some of the training focuses on the VP’s outgroups.


\subsection{Vice-President Events}\label{vice-president-events}

The Vice-President Events will manage all events or delegate
responsibility to manage events carried out by CSUS, should an event not
be handled by another member of CSUS, minimise cost of events, and
promote events.

\subsection{Meetings of the Executive
Council}\label{meetings-of-the-executive-council}

Quorum for a Regular meeting of CSUS Executive Council is four members
and quorum for online interactions, where relevant, is six members. Each
member of the Executive Council has a single vote, and the president
decides the course of action in the instance of a tie. The council may
run its meetings as deemed appropriate, all of which are available to
the public.

\subsection{Powers of Assembly}\label{powers-of-assembly}

The General Assembly may make any decision, including the ratification
or rejection of any Executive Council decision and initiation of
impeachment.

\subsection{Meetings of the General
Assembly}\label{meetings-of-the-general-assembly}

\begin{enumerate}
\def\labelenumi{\arabic{enumi}.}
\item
  The executive council may call a general assembly at any time.
\item
  Students should be notified of a general assembly at least 5 days
  prior the assembly.
\item
  There are no restrictions or regulations regarding the motions at the
  general assembly.
\item
  Students should be notified of the agenda at least 2 days prior to the
  assembly.
\item
  The Quorum for a GA is 20\% of the membership.
\item
  Procedures at a GA are as seen fit by the current council.
\item
  General assemblies are held at the discretion of the current council,
  but if at least 10\% of students request a GA one must be held within
  20 days of the request.
\end{enumerate}

\subsection{Committees of CSUS}\label{committees-of-csus}

\begin{enumerate}
\def\labelenumi{\arabic{enumi}.}
\item
  CSUS may create committees and subgroups as needed.
\item
  A committee's bylaws are initially created by CSUS and students
  interested in forming this committee and a 3/5 majority vote from the
  executive council modifies a committe's bylaws or existencee.
\end{enumerate}

\subsection{Electoral Officer}\label{electoral-officer}

\begin{enumerate}
\def\labelenumi{\arabic{enumi}.}
\item
  The Chief Returning Officer is responsible for administering
  elections.
\item
  The CRO may not be a candidate in any Society election. The Executive
  Council elects a new CRO if the CRO decides to run for a position.
\end{enumerate}

\subsection{Referenda}\label{referenda}

\begin{enumerate}
\def\labelenumi{\arabic{enumi}.}
\item
  A referendum is initiated by petition of 10\% of the society or by 2/3
  majority on council.
\item
  Referendums must be publicly available 6 days before vote and all
  details of the voting location, hours, etc. must also be posted.
\item
  The Referendum shall be considered valid only if a minimum of thirty
  percent (30\%) of CSUS members vote.
\item
  Referendums are passed by a majority of membership.
\item
  Referendums overrule decisions by council or by general assembly.
\end{enumerate}

\subsection{Impeachment}\label{impeachment}

Council members are impeached by 2/3 vote of executive council or by a
25\% vote of constituency.

\section{Elections}\label{elections}

\subsection{Eligible Voters and
Candidates}\label{eligible-voters-and-candidates}

\begin{enumerate}
\def\labelenumi{\arabic{enumi}.}
\item
  CSUS runs elections for the following positions: President, Vice
  President (VP) Internal, VP External, VP Finance, VP Academic, VP
  Administration, VP Diversity, VP Sustainability, and VP Events.
\item
  A U1 Representative, which can be more than one representative at the
  choice of the elected representative, is elected by the regular
  members who are first year computer science students, including those
  who have just transferred. The U1 Representative(s) have one vote on
  executive council.
\end{enumerate}

\subsection{Procedures}\label{procedures}

Elections are held between the first and last days of March excepting
the U1 rep, who is appointed between the first and last days of
September. Vacancies are filled by elections at the discretion of the
CRO. Elections can also be initiated and administered collectively by
the executive committee by a 2/3 vote to fill vacancies and such
elections will run for fifteen days starting from the day of the vote.

\subsection{Terms of Office}\label{terms-of-office}

The terms for members of the Executive Council are from the April 15 and
last one year excepting the U1 rep, who starts on the day of election or
selection.

\subsection{Order of Succession}\label{order-of-succession}

If the need for succession arises, another executive may assume his or
her duties by a 2/3 election by the executive council or, should a 2/3
vote not be reached within 5 days, a reelection from the first day at
the end of those 5 days until 10 days later.

\section{The Constitution}\label{the-constitution}

This Constitution supersedes and repeals all previous Constitutions. The
Constitution of the society may only be amended by a referendum with a
majority of two-thirds (2/3) of the voting members voting in favour.
This constitutions comes into force immediately upon approval.

\section{Bylaws}\label{bylaws}

\subsection{Nomination Rules}\label{nomination-rules}

\begin{enumerate}
\def\labelenumi{\arabic{enumi}.}
\item
  The CRO posts a list of positions open to nomination and election
  along with a time for the opening and closing of the nominations.
\item
  All nominations must be signed by ten students eligible to vote
  excepting for the U1 elections, for which five students must nominate.
\item
  The CRO validates the submitted nominations and publicises them within
  twelve hours of the closing of the nominations.
\end{enumerate}

\subsection{Campaigning Rules}\label{campaigning-rules}

The CRO may designate rules for the campaigning period. Candidates will
not be reimbursed.

\subsection{Balloting Rules}\label{balloting-rules}

The CRO shall be responsible for appointing polling clerks who will be
paid minimum wage by council or more by 2/3 vote. The CRO must cast a
sealed vote given to the council which will break ties.

\subsection{Count, Recount and
Protests}\label{count-recount-and-protests}

\begin{enumerate}
\def\labelenumi{\arabic{enumi}.}
\item
  Ballots are counted as soon as practicable after the closing of the
  polls under the supervision of the CRO and those whom the CRO
  designates to assist him.
\item
  No ballot shall be counted in the presence of less than two persons
  and may be rejected if deemed spoiled by CRO with a 2/3 vote by
  executive council.
\item
  All complaints, protest or petitions for a recount must be made to the
  CRO no later than three school days following the closing of the
  polls.
\end{enumerate}

\subsection{Invalidation}\label{invalidation}

The CRO shall invalidate the election at his discretion with a 2/3 vote
of council.

\subsection{Bylaw Amendments}\label{bylaw-amendments}

Amendments to the by-laws may be made at any meeting of CSUS Executive
Council and must be approved by two-thirds (2/3) of those present and
voting.
